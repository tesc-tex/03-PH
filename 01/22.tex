\subsection{Дивергенция векторной функции. Теорема Гаусса в дифференциальной форме}

Теорема Гаусса выражает замечательное свойство электрического поля, которое позволяет представить эту теорему в иной форме, расширяющей ее возможности как инструмента исследования и расчета. Найдем дифференциальную форму теоремы Гаусса, в которой устанавливается связь между объемной плотностью заряда p и *изменениями напряженности* (E) *в окрестности данной точки пространства.*

Пусть имеем заряд q в объеме V, охватываемом замкнутой поверхностью S, представим его как $q_{внут}=<\rho>V$, где $<\rho>$ — среднее по объему $V$ значение объемной плотности заряда.

\begin{theorem}
    Теорема Гаусса:

    $$
    Ф_E=\oint_S\vec Ed\vec S=\frac{q}{\varepsilon_0}
    $$
\end{theorem}

Подставим его в уравнение заряда и разделим на $V$:

$$\frac{1}{V}\oint_S\vec Ed\vec S=\frac{<\rho>}{\varepsilon_0}$$

Устремим объем к 0, стягивая к интересующей нас точке поля, тогда $<\rho>$ будет стремиться к значению $\rho$ в данной точке поля, 
а левая часть будет стремиться к $\frac{\rho}{\varepsilon_0}$.

Величина, являющаяся таким пределом отношения называется дивергенцией поля и обозначается как:

$$
div\vec E=\lim_{V\to0}\bigg(\frac{1}{V}\oint_S\vec Ed\vec S\bigg)
$$

$$
div\vec E= \frac{\partial E_x}{\partial x} + \frac{\partial E_y}{\partial y} + \frac{\partial E_z}{\partial z}
$$

$$
div\vec E=\frac{\rho}{\varepsilon_0}
$$

оно и выражает теорему Гаусса в дифференциальной форме.