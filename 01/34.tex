\subsection{Включение и отключение конденсатора от источника постоянной ЭДС}

Цепи.

Через конденсатор постоянный ток не течет. Напряжение на параллельных участках цепи одинаково. 
В системе отключенных конденсаторов заряд всегда остается постоянным. Напряжение и ёмкость может меняться.

Выделившееся количество теплоты равно разности начальной и конечной энергии:

$$
Q=E_н-E_к
$$

Начальные и конечные энергии определяются энергиями конденсаторов и катушек индуктивности, входящих в цепь.

После установления равновесия, напряжение есть только на конденсаторах, не подключенных параллельно к резисторам.
\begin{definition}
    Конденсатор в цепи постоянного тока.

    Плоский конденсатор представляет собой пластинки, на которых может скапливаться заряд. 
    Между пластинками находится пространство, заполненное диэлектриком. Поскольку диэлектрики — вещества, 
    плохо проводящие ток, от одной пластины конденсатора через слой диэлектрика на другую пластину заряд 
    перейти не может, а значит, через конденсатор ток не проходит. Если на участке цепи находится такой конденсатор 
    — этот участок “заблокирован”, тока в нем нет.
\end{definition}

Если на участке цепи находится конденсатор не заряженный (или заряженный частично), а цепь подключают к источнику тока 
— на обкладках конденсатора начинает скапливаться заряд. Это означает, что на этом участке цепи до конденсатора есть 
ток — до тех пор, пока конденсатор не заряжен полностью.

\begin{remark}
    Если цепь отключить от источника тока, и в ней есть заряженный конденсатор — конденсатор начинает разряжаться.
\end{remark}

Заряды с одной обкладки конденсатора пытаются перейти на другую, по “длинному пути” - через всю цепь, создавая таким 
образом ток. Ток в такой цепи будет до тех пор, пока конденсатор не разрядится.

Напряжения на всех параллельных участках цепи равны — это основное свойство параллельного подключения. 
Вне зависимости от того, находится на ветви резистор или конденсатор.

Благодаря этому свойству, зная, например, энергию, скопившуюся на заряженном конденсаторе, или его заряд, можно 
вычислить напряжение на резисторах.

\begin{definition}
    Заряженный конденсатор, отключенный от цепи.

    У заряженного конденсатора на обкладках находится определенное количество заряда. 
    Если конденсатор отключить от цепи — заряду некуда переместиться, и он остается на конденсаторе неизменным. 
    Получить дополнительный заряд, если он заряжен не до конца, конденсатору тоже неоткуда. Заряд конденсатора, 
    отключенного от цепи, постоянен.
\end{definition}

В цепи энергия скапливается на конденсатора (энергия электрического поля) и на катушках индуктивности 
(энергия магнитного поля). Поэтому энергия электромагнитных сил в цепи в любой момент равна сумме энергия на 
конденсаторах и на катушках, которые входят в цепи.

Энергия электрического поля заряженного конденсатора:

$$
W_e=\frac{CU^2}{2}=\frac{qU}{2}=\frac{q^2}{2C}
$$

где $W_e$, $[Дж]$ - энергия электрического поля конденсатора,
$C$, $[Ф]$ - электроемкость конденсатора,
$U$, $[В]$ - напряжение на обкладках конденсатора,
$q$, $[Кл]$ - заряд на обкладках конденсатора

Энергия магнитного поля катушки индуктивности:

$$
E=\frac{LI^2}{2}
$$

где $E$, $[Дж]$ - энергия магнитного поля катушки,
$L$, $[Гн]$ - индуктивность катушка,
$I$, $[А]$ - сила тока в катушке