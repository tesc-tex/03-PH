\subsection{Ротор векторной функции. Физический смысл ротора. Теорема Стокса}

\begin{definition}
    Ротор векторного поля — вектор, проекция которого $rot_n$ на каждое направление $n$ есть предел отношения циркуляции векторного поля 
    по контуру $L$, являющемуся краем плоской площадки $\Delta S$, перпендикулярной этому направлению, 
    к величине этой площадки (площади), когда размеры площадки стремятся к 0, а сама площадка стягивается в точку:

    $$
    rot_na=\lim_{\Delta S\to0}\frac{\oint_La\cdot dr}{\Delta S}
    $$
\end{definition}

\begin{remark}
    Физический смысл.

    Поверхностная плотность циркуляции поля по бесконечно малой окружности равна проекции ротора на нормаль к данной окружности.

    Указывает направление, ортогонально которому вращательная способность поля наибольшая.
\end{remark}

\begin{theorem}
    Теорема Стокса.

    Циркуляция вектора $\overline a(т.M)=P(x,y,z)i+Q(x,y,z)j+R(x,y,z)k$ по замкнутому контуру равна потоку ротора этого вектора 
    через любую поверхность, ограниченную этим контуром (натянутую на этот контур)
    
    $$
    (rotF\cdot n)(M)=\lim_{Г\to M}\frac{\oint_{\partial Г}F\ dl}{S(Г)}
    $$

    где $Г$ — кусок поверхности, стягивающийся к точке $M$,
    $S(Г)$ — площадь $Г$,
    $n$ — единичный вектор нормали к поверхности $Г$, определяющий её ориентацию.
\end{theorem}